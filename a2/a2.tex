\documentclass[fontsize=11pt]{article}
\usepackage{amsmath}
\usepackage[utf8]{inputenc}
\usepackage[margin=0.75in]{geometry}

\title{CSC110 Fall 2021 Assignment 2: Logic, Constraints, and Nested Data}
\author{Nathaniel Yong}
\date{\today}

\begin{document}
\maketitle

\section*{Part 1: Predicate Logic}

\begin{enumerate}

\item[1.]
    \begin{enumerate}
        \item[1.] Yes, $D_1$ could be the set of natural numbers. Statement 1 is true but statement 2 is false. For every $x$, you can take $y = x + 1$ and statement 1 is satisfied. However, statement 2 is false for $y = 0$, because there is obviously no natural number $x$ where $x < 0$.
        \item[2.] Yes, $D_2$ could be the set of all integers. For every $x$ in this set, you can take $y = x + 1$, and statement 1 is satisfied. For every $y$ in this set, you can take $x = y - 1$, and statement 2 is satisfied.
        \item[3.] Yes, let $D_3=\{0\}$. When $x = 0$, there is no $y$ less than it. When $y = 0$, there is no $x$ bigger than it.
    \end{enumerate}

\item[2.]
    \begin{enumerate}
        \item[1.] $P(x): x > 2$, where $x\in S$
        \item[2.] $Q(x): x > 1$, where $x\in S$
    \end{enumerate}
    These two predicates make statement 3 false and statement 4 true. Statement 3 is false, because if you take $x = 0$, $P(x)\land Q(x)$ evaluates to False $\land$ False, which is simply False.  Statement 4 is true, because if $x\leq2$, then $P(x)$ is false, and so the statement is vacuously true. However, if $x > 2$, then $P(x)$ is true, and $Q(x)$ must also be true because if $x > 2$ then obviously $x > 1$.

\item[3.]
Complete this part in the provided \texttt{a2\_part1.py} starter file.
Do \textbf{not} include your solution in this file.

\item[4.]
Complete this part in the provided \texttt{a2\_part1.py} starter file.
Do \textbf{not} include your solution in this file.

\end{enumerate}

\section*{Part 2: Conditional Execution}

Complete this part in the provided \texttt{a2\_part2.py} starter file.
Do \textbf{not} include your solution in this file.

\newpage

\section*{Part 3: Generating a Timetable}

\begin{enumerate}

\item[1.]
Complete this part in the provided \texttt{a2\_part3.py} starter file.
Do \textbf{not} include your solution in this file.

\item[2.]

\begin{enumerate}
\item[(a)]

\emph{IMPORTANT DEFINITIONS/NOTATION} (don't change this text!)

We define the following sets:

\begin{itemize}
\item $C$: the set of all possible courses
\item $S$: the set of all possible sections
\item $M$: the set of all possible meeting times
\item $SC$: the set of all possible schedules
\end{itemize}

We also define the following notation for expressions involving the elements of these sets:

\begin{itemize}
\item
The first three (courses/sections/meeting times) are represented as tuples (as described in the assignment handout), and you can use the indexing operation on these values. For example, you could translate ``every section term is in $\{'F', 'S', 'Y'\}$'' into predicate logic as the statement:

    \[\forall s \in S,~ s[1] \in \{'F', 'S', 'Y' \} \]

\item
The start and end times of a meeting time can be compared chronologically using the standard $<$, $\leq$, $>$, and $\geq$ operators.

\item
For a section $s \in S$, $s[2]$ represents a tuple of meeting times.
You may use standard set operations and quantifiers for these tuples (pretend they are sets).
For example, we can say:

    \begin{itemize}
    \item $\forall s \in S,~ s[2] \subseteq M$
    \item $\forall s \in S,~ \forall m \in s[2],~ m[1] < m[2]$
    \end{itemize}

\item
Finally, for a schedule $sc \in SC$, you can use the notation $sc.sections$ to refer to a set of all sections in that schedule.
You can use quantifiers with that set of schedules as well, e.g.
$\forall s \in sc.sections,~ ...$
\end{itemize}

\textbf{Predicate for meeting times conflicting:}
% TODO: fill in the predicate definition for two meeting times conflicting

\begin{align*}
MeetingTimesConflict(m_1, m_2) : (m_1[0] = m_2[0]) \land (m_1[1]\leq m_2[1]<m_1[2] \lor m_1[1] < m_2[2]\leq m_1[2]) \\
\qquad \text{where $m_1, m_2 \in M$}
\end{align*}

\smallskip

\textbf{Predicate for sections conflicting:}
% TODO: fill in the predicate definition for two sections conflicting.
% Use the MeetingTimesConflict predicate in your response.

\begin{align*}
SectionsConflict(s_1, s_2) : (s_1[1] = s_2[1] \lor s_1[1]=`Y' \lor s_2[1]=`Y') \land \\ (\exists m_1,m_2 \in s_1[2], MeetingTimesConflict(m_1, m_2))
\qquad \text{where $s_1, s_2 \in S$}
\end{align*}

\smallskip

\textbf{Predicate for valid schedule:}
% TODO: fill in the predicate definition for a schedule being valid.
% Use the SectionsConflict predicate in your response.

\begin{align*}
IsValidSchedule(sc) : \forall s_1 \in sc.sections, \forall s_2 \in sc.sections, (s_1 \neq s_2 \implies \lnot SectionsConflict(s_1, s_2)) \\
\qquad \text{where $sc \in SC$}
\end{align*}


\item[(b)]
Complete this part in the provided \texttt{a2\_part3.py} starter file.
Do \textbf{not} include your solution in this file.
\end{enumerate}

\item[3.]

\begin{enumerate}
\item[(a)]

You may use all notation from question 2(a).
Note that a course $c \in C$ is a tuple, and $c[2]$ is a set of sections, and so can be quantified over: $\forall s \in c[2], ...$.

\smallskip

\textbf{Predicate for section-schedule compatibility:}
% TODO: fill in the predicate definition for a section being compatible with a schedule.

\begin{align*}
IsCompatibleSection(sc, s) : \forall s_2 \in sc.sections, \lnot SectionsConflict(s, s_2)
\qquad \text{where $sc \in SC, s \in S$}
\end{align*}

\smallskip

\textbf{Predicate for course-schedule compatibility:}
% TODO: fill in the predicate definition for a course being compatible with a schedule.
% Use IsCompatibleSection in your response.

\begin{align*}
IsCompatibleCourse(sc, c) : \exists s \in c[2], IsCompatibleSection(sc, s)
\qquad \text{where $sc \in SC, c \in C$}
\end{align*}

\item[(b)]
Complete this part in the provided \texttt{a2\_part3.py} starter file.
Do \textbf{not} include your solution in this file.
\end{enumerate}

\end{enumerate}

\section*{Part 4: Processing Raw Data}
Complete this part in the provided \texttt{a2\_part4.py} starter file.
Do \textbf{not} include your solution in this file.

\end{document}
